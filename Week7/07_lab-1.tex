% Options for packages loaded elsewhere
\PassOptionsToPackage{unicode}{hyperref}
\PassOptionsToPackage{hyphens}{url}
%
\documentclass[
]{article}
\usepackage{lmodern}
\usepackage{amssymb,amsmath}
\usepackage{ifxetex,ifluatex}
\ifnum 0\ifxetex 1\fi\ifluatex 1\fi=0 % if pdftex
  \usepackage[T1]{fontenc}
  \usepackage[utf8]{inputenc}
  \usepackage{textcomp} % provide euro and other symbols
\else % if luatex or xetex
  \usepackage{unicode-math}
  \defaultfontfeatures{Scale=MatchLowercase}
  \defaultfontfeatures[\rmfamily]{Ligatures=TeX,Scale=1}
\fi
% Use upquote if available, for straight quotes in verbatim environments
\IfFileExists{upquote.sty}{\usepackage{upquote}}{}
\IfFileExists{microtype.sty}{% use microtype if available
  \usepackage[]{microtype}
  \UseMicrotypeSet[protrusion]{basicmath} % disable protrusion for tt fonts
}{}
\makeatletter
\@ifundefined{KOMAClassName}{% if non-KOMA class
  \IfFileExists{parskip.sty}{%
    \usepackage{parskip}
  }{% else
    \setlength{\parindent}{0pt}
    \setlength{\parskip}{6pt plus 2pt minus 1pt}}
}{% if KOMA class
  \KOMAoptions{parskip=half}}
\makeatother
\usepackage{xcolor}
\IfFileExists{xurl.sty}{\usepackage{xurl}}{} % add URL line breaks if available
\IfFileExists{bookmark.sty}{\usepackage{bookmark}}{\usepackage{hyperref}}
\hypersetup{
  pdftitle={Correlation},
  pdfauthor={Karmanya Pathak},
  hidelinks,
  pdfcreator={LaTeX via pandoc}}
\urlstyle{same} % disable monospaced font for URLs
\usepackage[margin=1in]{geometry}
\usepackage{color}
\usepackage{fancyvrb}
\newcommand{\VerbBar}{|}
\newcommand{\VERB}{\Verb[commandchars=\\\{\}]}
\DefineVerbatimEnvironment{Highlighting}{Verbatim}{commandchars=\\\{\}}
% Add ',fontsize=\small' for more characters per line
\usepackage{framed}
\definecolor{shadecolor}{RGB}{248,248,248}
\newenvironment{Shaded}{\begin{snugshade}}{\end{snugshade}}
\newcommand{\AlertTok}[1]{\textcolor[rgb]{0.94,0.16,0.16}{#1}}
\newcommand{\AnnotationTok}[1]{\textcolor[rgb]{0.56,0.35,0.01}{\textbf{\textit{#1}}}}
\newcommand{\AttributeTok}[1]{\textcolor[rgb]{0.77,0.63,0.00}{#1}}
\newcommand{\BaseNTok}[1]{\textcolor[rgb]{0.00,0.00,0.81}{#1}}
\newcommand{\BuiltInTok}[1]{#1}
\newcommand{\CharTok}[1]{\textcolor[rgb]{0.31,0.60,0.02}{#1}}
\newcommand{\CommentTok}[1]{\textcolor[rgb]{0.56,0.35,0.01}{\textit{#1}}}
\newcommand{\CommentVarTok}[1]{\textcolor[rgb]{0.56,0.35,0.01}{\textbf{\textit{#1}}}}
\newcommand{\ConstantTok}[1]{\textcolor[rgb]{0.00,0.00,0.00}{#1}}
\newcommand{\ControlFlowTok}[1]{\textcolor[rgb]{0.13,0.29,0.53}{\textbf{#1}}}
\newcommand{\DataTypeTok}[1]{\textcolor[rgb]{0.13,0.29,0.53}{#1}}
\newcommand{\DecValTok}[1]{\textcolor[rgb]{0.00,0.00,0.81}{#1}}
\newcommand{\DocumentationTok}[1]{\textcolor[rgb]{0.56,0.35,0.01}{\textbf{\textit{#1}}}}
\newcommand{\ErrorTok}[1]{\textcolor[rgb]{0.64,0.00,0.00}{\textbf{#1}}}
\newcommand{\ExtensionTok}[1]{#1}
\newcommand{\FloatTok}[1]{\textcolor[rgb]{0.00,0.00,0.81}{#1}}
\newcommand{\FunctionTok}[1]{\textcolor[rgb]{0.00,0.00,0.00}{#1}}
\newcommand{\ImportTok}[1]{#1}
\newcommand{\InformationTok}[1]{\textcolor[rgb]{0.56,0.35,0.01}{\textbf{\textit{#1}}}}
\newcommand{\KeywordTok}[1]{\textcolor[rgb]{0.13,0.29,0.53}{\textbf{#1}}}
\newcommand{\NormalTok}[1]{#1}
\newcommand{\OperatorTok}[1]{\textcolor[rgb]{0.81,0.36,0.00}{\textbf{#1}}}
\newcommand{\OtherTok}[1]{\textcolor[rgb]{0.56,0.35,0.01}{#1}}
\newcommand{\PreprocessorTok}[1]{\textcolor[rgb]{0.56,0.35,0.01}{\textit{#1}}}
\newcommand{\RegionMarkerTok}[1]{#1}
\newcommand{\SpecialCharTok}[1]{\textcolor[rgb]{0.00,0.00,0.00}{#1}}
\newcommand{\SpecialStringTok}[1]{\textcolor[rgb]{0.31,0.60,0.02}{#1}}
\newcommand{\StringTok}[1]{\textcolor[rgb]{0.31,0.60,0.02}{#1}}
\newcommand{\VariableTok}[1]{\textcolor[rgb]{0.00,0.00,0.00}{#1}}
\newcommand{\VerbatimStringTok}[1]{\textcolor[rgb]{0.31,0.60,0.02}{#1}}
\newcommand{\WarningTok}[1]{\textcolor[rgb]{0.56,0.35,0.01}{\textbf{\textit{#1}}}}
\usepackage{graphicx,grffile}
\makeatletter
\def\maxwidth{\ifdim\Gin@nat@width>\linewidth\linewidth\else\Gin@nat@width\fi}
\def\maxheight{\ifdim\Gin@nat@height>\textheight\textheight\else\Gin@nat@height\fi}
\makeatother
% Scale images if necessary, so that they will not overflow the page
% margins by default, and it is still possible to overwrite the defaults
% using explicit options in \includegraphics[width, height, ...]{}
\setkeys{Gin}{width=\maxwidth,height=\maxheight,keepaspectratio}
% Set default figure placement to htbp
\makeatletter
\def\fps@figure{htbp}
\makeatother
\setlength{\emergencystretch}{3em} % prevent overfull lines
\providecommand{\tightlist}{%
  \setlength{\itemsep}{0pt}\setlength{\parskip}{0pt}}
\setcounter{secnumdepth}{-\maxdimen} % remove section numbering

\title{Correlation}
\author{Karmanya Pathak}
\date{2020-03-29}

\begin{document}
\maketitle

\emph{Title}: Big Data Analytics Services for Enhancing Business
Intelligence

\emph{Abstract}: This article examines how to use big data analytics
services to enhance business intelligence (BI). More specifically, this
article proposes an ontology of big data analytics and presents a big
data analytics service-oriented architecture (BASOA), and then applies
BASOA to BI, where our surveyed data analysis shows that the proposed
BASOA is viable for enhancing BI and enterprise information systems.
This article also explores temporality, expectability, and relativity as
the characteristics of intelligence in BI. These characteristics are
what customers and decision makers expect from BI in terms of systems,
products, and services of organizations. The proposed approach in this
article might facilitate the research and development of business
analytics, big data analytics, and BI as well as big data science and
big data computing.

\hypertarget{dataset}{%
\section{Dataset:}\label{dataset}}

\begin{verbatim}
-   Gender of the participant surveyed on these topics
-   Temporality: an average score of the rated ability to adapt to change over time 1 (not changing) to 7 (changing a lot)
-   Expectability: a rated degree of satisfaction with the BI
-   Relativity: average score rating of how much better one system is than another in BI 1 (not very good) to 7 (very good)
-   Positive emotion: how positive participants felt about BI (higher scores are more positive, ranges from 1 to 7).
\end{verbatim}

\begin{Shaded}
\begin{Highlighting}[]
\NormalTok{df1 <-}\StringTok{ }\KeywordTok{read.csv}\NormalTok{(}\StringTok{"D:}\CharTok{\textbackslash{}\textbackslash{}}\StringTok{Harrisburg}\CharTok{\textbackslash{}\textbackslash{}}\StringTok{Lectures}\CharTok{\textbackslash{}\textbackslash{}}\StringTok{Spring 2020}\CharTok{\textbackslash{}\textbackslash{}}\StringTok{ANLY 500 - Principles of Analytics I}\CharTok{\textbackslash{}\textbackslash{}}\StringTok{Assignment 7}\CharTok{\textbackslash{}\textbackslash{}}\StringTok{07_data.csv"}\NormalTok{)}

\NormalTok{df2 <-}\StringTok{ }\NormalTok{df1}
\end{Highlighting}
\end{Shaded}

\hypertarget{data-screening}{%
\section{Data Screening:}\label{data-screening}}

\hypertarget{accuracy}{%
\subsection{Accuracy:}\label{accuracy}}

\begin{verbatim}
a.  Include output that indicates if the data are or are not accurate.
b.  If the data are not accurate, delete the inaccurate scores.
c.  Include a summary that shows that you fixed the inaccurate scores.
\end{verbatim}

\begin{Shaded}
\begin{Highlighting}[]
\KeywordTok{str}\NormalTok{(df2)}
\end{Highlighting}
\end{Shaded}

\begin{verbatim}
## 'data.frame':    300 obs. of  5 variables:
##  $ gender       : Factor w/ 2 levels "men","women": 2 2 2 2 2 2 2 2 2 2 ...
##  $ temporality  : num  4.7 4.22 4.59 4.33 4.26 ...
##  $ expectability: int  6 5 3 1 2 5 5 0 5 6 ...
##  $ relativity   : num  0.591 4.399 2.573 1.347 2.447 ...
##  $ positive     : num  3.698 -0.226 1.224 1.795 2.202 ...
\end{verbatim}

\begin{Shaded}
\begin{Highlighting}[]
\KeywordTok{summary}\NormalTok{(df2)}
\end{Highlighting}
\end{Shaded}

\begin{verbatim}
##    gender     temporality    expectability     relativity        positive      
##  men  :150   Min.   :1.737   Min.   :0.000   Min.   :-2.301   Min.   :-0.9128  
##  women:150   1st Qu.:2.823   1st Qu.:2.000   1st Qu.: 2.439   1st Qu.: 2.0175  
##              Median :3.581   Median :3.000   Median : 3.564   Median : 3.0780  
##              Mean   :3.532   Mean   :3.643   Mean   : 3.569   Mean   : 3.1446  
##              3rd Qu.:4.225   3rd Qu.:5.000   3rd Qu.: 4.731   3rd Qu.: 4.2614  
##              Max.   :5.184   Max.   :9.000   Max.   :10.508   Max.   : 8.2126  
##              NA's   :9       NA's   :9       NA's   :9        NA's   :9
\end{verbatim}

\begin{Shaded}
\begin{Highlighting}[]
\NormalTok{df2}\OperatorTok{$}\NormalTok{expectability[ df2}\OperatorTok{$}\NormalTok{expectability }\OperatorTok{<}\StringTok{ }\DecValTok{1}\NormalTok{ ] =}\StringTok{ }\OtherTok{NA}
\NormalTok{df2}\OperatorTok{$}\NormalTok{expectability[ df2}\OperatorTok{$}\NormalTok{expectability }\OperatorTok{>}\StringTok{ }\DecValTok{7}\NormalTok{ ] =}\StringTok{ }\OtherTok{NA}

\CommentTok{#table(df$relativity)}

\NormalTok{df2}\OperatorTok{$}\NormalTok{relativity[ df2}\OperatorTok{$}\NormalTok{relativity }\OperatorTok{<}\StringTok{ }\DecValTok{1}\NormalTok{ ] =}\StringTok{ }\OtherTok{NA}
\NormalTok{df2}\OperatorTok{$}\NormalTok{relativity[ df2}\OperatorTok{$}\NormalTok{relativity }\OperatorTok{>}\StringTok{ }\DecValTok{7}\NormalTok{ ] =}\StringTok{ }\OtherTok{NA}

\CommentTok{#table(df$relativity)}

\NormalTok{df2}\OperatorTok{$}\NormalTok{positive[ df2}\OperatorTok{$}\NormalTok{positive }\OperatorTok{<}\StringTok{ }\DecValTok{1}\NormalTok{ ] =}\StringTok{ }\OtherTok{NA}
\NormalTok{df2}\OperatorTok{$}\NormalTok{positive[ df2}\OperatorTok{$}\NormalTok{positive }\OperatorTok{>}\StringTok{ }\DecValTok{7}\NormalTok{ ] =}\StringTok{ }\OtherTok{NA}
\CommentTok{#table(df$positive)}
\KeywordTok{summary}\NormalTok{(df2)}
\end{Highlighting}
\end{Shaded}

\begin{verbatim}
##    gender     temporality    expectability     relativity       positive    
##  men  :150   Min.   :1.737   Min.   :1.000   Min.   :1.065   Min.   :1.053  
##  women:150   1st Qu.:2.823   1st Qu.:2.000   1st Qu.:2.625   1st Qu.:2.337  
##              Median :3.581   Median :3.000   Median :3.618   Median :3.250  
##              Mean   :3.532   Mean   :3.633   Mean   :3.650   Mean   :3.423  
##              3rd Qu.:4.225   3rd Qu.:5.000   3rd Qu.:4.676   3rd Qu.:4.396  
##              Max.   :5.184   Max.   :7.000   Max.   :6.952   Max.   :6.918  
##              NA's   :9       NA's   :22      NA's   :44      NA's   :41
\end{verbatim}

\hypertarget{missing}{%
\subsection{Missing:}\label{missing}}

\begin{verbatim}
a.  Since any accuracy errors will create more than 5% missing data, exclude all data pairwise for the rest of the analyses. 
\end{verbatim}

\begin{Shaded}
\begin{Highlighting}[]
\NormalTok{no_miss <-}\StringTok{ }\KeywordTok{subset}\NormalTok{(df2, }\KeywordTok{is.na}\NormalTok{(df2}\OperatorTok{$}\NormalTok{temporality)}\OperatorTok{==}\OtherTok{FALSE} \OperatorTok{&}\StringTok{ }\KeywordTok{is.na}\NormalTok{(df2}\OperatorTok{$}\NormalTok{expectability)}\OperatorTok{==}\OtherTok{FALSE} \OperatorTok{&}\StringTok{ }\KeywordTok{is.na}\NormalTok{(df2}\OperatorTok{$}\NormalTok{relativity)}\OperatorTok{==}\OtherTok{FALSE} \OperatorTok{&}\StringTok{ }\KeywordTok{is.na}\NormalTok{(df2}\OperatorTok{$}\NormalTok{positive)}\OperatorTok{==}\OtherTok{FALSE}\NormalTok{)}
\KeywordTok{summary}\NormalTok{(no_miss)}
\end{Highlighting}
\end{Shaded}

\begin{verbatim}
##    gender     temporality    expectability     relativity       positive    
##  men  :106   Min.   :1.737   Min.   :1.000   Min.   :1.065   Min.   :1.053  
##  women: 93   1st Qu.:2.849   1st Qu.:3.000   1st Qu.:2.634   1st Qu.:2.311  
##              Median :3.508   Median :3.000   Median :3.567   Median :3.232  
##              Mean   :3.491   Mean   :3.593   Mean   :3.636   Mean   :3.417  
##              3rd Qu.:4.164   3rd Qu.:5.000   3rd Qu.:4.684   3rd Qu.:4.400  
##              Max.   :5.102   Max.   :7.000   Max.   :6.891   Max.   :6.918
\end{verbatim}

\hypertarget{outliers}{%
\subsection{Outliers:}\label{outliers}}

\begin{verbatim}
a.  Include a summary of your mahal scores.
b.  What are the df for your Mahalanobis cutoff?
c.  What is the cut off score for your Mahalanobis measure?
d.  How many outliers did you have? 
\end{verbatim}

\begin{Shaded}
\begin{Highlighting}[]
\CommentTok{#a.}
\KeywordTok{str}\NormalTok{(no_miss)}
\end{Highlighting}
\end{Shaded}

\begin{verbatim}
## 'data.frame':    199 obs. of  5 variables:
##  $ gender       : Factor w/ 2 levels "men","women": 2 2 2 2 2 2 2 2 2 2 ...
##  $ temporality  : num  4.59 4.33 4.26 4.66 4.04 ...
##  $ expectability: int  3 1 2 5 5 5 6 5 7 5 ...
##  $ relativity   : num  2.57 1.35 2.45 3.78 5.45 ...
##  $ positive     : num  1.22 1.8 2.2 2.55 3.02 ...
\end{verbatim}

\begin{Shaded}
\begin{Highlighting}[]
\NormalTok{no_miss}\OperatorTok{$}\NormalTok{gender <-}\StringTok{ }\KeywordTok{as.numeric}\NormalTok{(no_miss}\OperatorTok{$}\NormalTok{gender)}
\NormalTok{mahal =}\StringTok{ }\KeywordTok{mahalanobis}\NormalTok{(no_miss,}
                    \KeywordTok{colMeans}\NormalTok{(no_miss, }\DataTypeTok{na.rm=}\OtherTok{TRUE}\NormalTok{),}
                    \KeywordTok{cov}\NormalTok{(no_miss, }\DataTypeTok{use =}\StringTok{"pairwise.complete.obs"}\NormalTok{)}
\NormalTok{                    )}
\NormalTok{mahal}
\end{Highlighting}
\end{Shaded}

\begin{verbatim}
##         3         4         5         6         7         9        12        16 
##  5.173705  8.705403  4.113660  3.195317  2.760031  9.694104  4.796637  8.232600 
##        17        19        20        22        24        28        29        30 
##  6.989345  3.123559  3.713654 12.180095  8.777336  3.123600  4.592998  5.129798 
##        31        32        34        37        38        39        42        43 
##  8.870841  3.218623  7.902896  2.955135  3.175031  8.785114  7.548496  7.587460 
##        44        45        46        49        50        52        53        55 
##  3.884792  4.438407  4.050991  2.762530  4.094642  4.971903  3.535604  3.694822 
##        56        59        60        62        63        64        65        67 
##  2.072417  3.639987  9.192360  9.383698  1.814481  3.223031  4.204170  2.437363 
##        68        69        70        71        73        74        75        76 
##  3.573854 11.687956  3.116934  7.094872  6.959299  4.261758  1.670075  4.602651 
##        77        78        81        82        83        85        86        88 
##  3.771035  9.151379  4.267840  7.164489  8.442925  4.444914  2.463844  9.509837 
##        90        97        98       100       101       102       104       105 
##  5.749490  3.188488  5.167400  5.352611  3.784736  2.783307  6.221035  6.273785 
##       106       107       108       110       112       113       114       116 
##  6.614528  3.371546  5.143446  9.216289  4.164509  5.375734  4.951599  3.781767 
##       119       121       122       124       126       127       129       130 
##  6.563688  5.335984  7.041592  7.436183  7.248670  6.349478  2.761290  5.622344 
##       131       132       133       135       137       139       142       143 
##  6.402980  3.225894  4.998631  4.349487  5.842897  6.499433  6.609823  1.873188 
##       144       146       148       149       150       151       152       153 
##  2.374542  2.747883  2.837559  5.909568  3.939083  6.550338  1.416239  6.764271 
##       154       155       157       158       159       160       162       163 
##  1.744653 14.145021  7.292288  1.510605  8.911743  3.893296  3.371522  3.652287 
##       164       166       171       172       173       174       175       178 
##  5.119587  1.832034  4.450481  3.014248  5.248819  2.445432  5.331486  3.523170 
##       179       180       181       182       183       184       187       188 
##  6.203487  4.262792  2.931815  3.816309  6.978477  9.494469  7.282103  2.291328 
##       190       191       194       195       196       198       199       201 
##  9.209333  7.782472  2.958115  4.092585  5.964329  4.207668  4.995629  3.815049 
##       202       203       204       206       207       208       209       210 
##  7.930977  3.074316  7.849840  5.092080  1.318739  2.808612  6.541754  2.804183 
##       212       213       216       217       218       220       221       222 
##  5.718393  3.525487  1.259470  5.655145  3.041396  3.497104  7.154859 13.402188 
##       223       224       225       226       227       229       230       231 
##  5.621879  4.031839  2.232967  2.163585  5.652425  3.832274  5.036014  1.264779 
##       232       233       234       235       236       239       240       241 
##  3.790919  1.507569  2.905366  4.196152  5.473969  3.983441  3.905880  4.477096 
##       244       246       248       249       250       251       252       255 
##  3.836541  4.861460  2.904178  6.616769  6.027505  7.875058  3.685401  3.872168 
##       256       257       258       259       260       261       262       263 
##  6.451456  1.799545  5.121325  2.587768  6.479328  3.878655  2.595982  3.789359 
##       265       268       269       270       271       274       275       276 
##  2.637599  4.529783  4.139586  3.687803  3.382491  4.597964  5.062751  5.028404 
##       278       279       280       282       283       286       287       288 
##  8.154345  2.901044  6.454348  1.890022  3.076607 12.823742  6.058405  7.717686 
##       289       290       292       294       295       296       298 
##  3.303212  2.033539  4.775521  2.459799  5.453165  6.775175  2.379570
\end{verbatim}

\begin{Shaded}
\begin{Highlighting}[]
\CommentTok{#b.}
\CommentTok{#199 observations, so degrees of freedom is 198}


\CommentTok{#c.}
\NormalTok{cutoff =}\StringTok{ }\KeywordTok{qchisq}\NormalTok{(}\DecValTok{1}\FloatTok{-.001}\NormalTok{,}\KeywordTok{ncol}\NormalTok{(no_miss))}
\KeywordTok{print}\NormalTok{(cutoff)}
\end{Highlighting}
\end{Shaded}

\begin{verbatim}
## [1] 20.51501
\end{verbatim}

\begin{Shaded}
\begin{Highlighting}[]
\StringTok{"Cutoff is 20.51501"}
\end{Highlighting}
\end{Shaded}

\begin{verbatim}
## [1] "Cutoff is 20.51501"
\end{verbatim}

\begin{Shaded}
\begin{Highlighting}[]
\CommentTok{#d.}
\KeywordTok{summary}\NormalTok{(mahal }\OperatorTok{<}\StringTok{ }\NormalTok{cutoff)}
\end{Highlighting}
\end{Shaded}

\begin{verbatim}
##    Mode    TRUE 
## logical     199
\end{verbatim}

\hypertarget{assumptions}{%
\section{Assumptions:}\label{assumptions}}

\hypertarget{linearity}{%
\subsection{Linearity:}\label{linearity}}

\begin{verbatim}
a.  Include a picture that shows how you might assess multivariate linearity.
b.  Do you think you've met the assumption for linearity? 
\end{verbatim}

\begin{Shaded}
\begin{Highlighting}[]
\NormalTok{random =}\StringTok{ }\KeywordTok{rchisq}\NormalTok{(}\KeywordTok{nrow}\NormalTok{(no_miss), }\DecValTok{198}\NormalTok{)}
\NormalTok{fake =}\StringTok{ }\KeywordTok{lm}\NormalTok{(random}\OperatorTok{~}\NormalTok{., }\DataTypeTok{data =}\NormalTok{ no_miss)}
\KeywordTok{summary}\NormalTok{(fake)}
\end{Highlighting}
\end{Shaded}

\begin{verbatim}
## 
## Call:
## lm(formula = random ~ ., data = no_miss)
## 
## Residuals:
##     Min      1Q  Median      3Q     Max 
## -50.572 -13.753  -2.285  13.395  55.139 
## 
## Coefficients:
##               Estimate Std. Error t value Pr(>|t|)    
## (Intercept)   202.8651     9.1811  22.096   <2e-16 ***
## gender         10.8668     5.0691   2.144   0.0333 *  
## temporality    -6.6689     3.2866  -2.029   0.0438 *  
## expectability   2.3078     1.0463   2.206   0.0286 *  
## relativity     -1.4950     1.1281  -1.325   0.1867    
## positive        0.2511     1.0998   0.228   0.8196    
## ---
## Signif. codes:  0 '***' 0.001 '**' 0.01 '*' 0.05 '.' 0.1 ' ' 1
## 
## Residual standard error: 19.96 on 193 degrees of freedom
## Multiple R-squared:  0.04973,    Adjusted R-squared:  0.02511 
## F-statistic:  2.02 on 5 and 193 DF,  p-value: 0.0775
\end{verbatim}

\begin{Shaded}
\begin{Highlighting}[]
\NormalTok{standardized =}\StringTok{ }\KeywordTok{rstudent}\NormalTok{(fake)}
\KeywordTok{qqnorm}\NormalTok{(standardized)}
\KeywordTok{abline}\NormalTok{(}\DecValTok{0}\NormalTok{,}\DecValTok{1}\NormalTok{)}
\end{Highlighting}
\end{Shaded}

\includegraphics{07_lab-1_files/figure-latex/linearity-1.pdf}

\begin{Shaded}
\begin{Highlighting}[]
\CommentTok{#yes I think we've met the assumption for linearity, because the q-q plot follows along the abline with few outlines at both ends of the tail.}
\end{Highlighting}
\end{Shaded}

\hypertarget{normality}{%
\subsection{Normality:}\label{normality}}

\begin{verbatim}
a.  Include a picture that shows how you might assess multivariate normality.
b.  Do you think you've met the assumption for normality? 
\end{verbatim}

\begin{Shaded}
\begin{Highlighting}[]
\KeywordTok{library}\NormalTok{(moments)}
\CommentTok{#Normality}
\KeywordTok{skewness}\NormalTok{(no_miss)}
\end{Highlighting}
\end{Shaded}

\begin{verbatim}
##        gender   temporality expectability    relativity      positive 
##  0.1309329477 -0.0006409599  0.2893538674  0.0962008323  0.4066039422
\end{verbatim}

\begin{Shaded}
\begin{Highlighting}[]
\KeywordTok{kurtosis}\NormalTok{(no_miss)}
\end{Highlighting}
\end{Shaded}

\begin{verbatim}
##        gender   temporality expectability    relativity      positive 
##      1.017143      2.026042      2.359912      2.360471      2.329268
\end{verbatim}

\begin{Shaded}
\begin{Highlighting}[]
\KeywordTok{hist}\NormalTok{(standardized, }\DataTypeTok{breaks =} \DecValTok{20}\NormalTok{)}
\end{Highlighting}
\end{Shaded}

\includegraphics{07_lab-1_files/figure-latex/normality-1.pdf}

\begin{Shaded}
\begin{Highlighting}[]
\CommentTok{#this is not completely normal, slightly skewed}
\end{Highlighting}
\end{Shaded}

\hypertarget{homogeneity-and-homoscedasticity}{%
\subsection{Homogeneity and
Homoscedasticity:}\label{homogeneity-and-homoscedasticity}}

\begin{verbatim}
a.  Include a picture that shows how you might assess multivariate homogeneity.
b.  Do you think you've met the assumption for homogeneity?
c.  Do you think you've met the assumption for homoscedasticity?
\end{verbatim}

\begin{Shaded}
\begin{Highlighting}[]
\CommentTok{#Homogeneity, Homoscedasticity}

\CommentTok{#Making the fit values standardized}
\NormalTok{fitvalues =}\StringTok{ }\KeywordTok{scale}\NormalTok{(fake}\OperatorTok{$}\NormalTok{fitted.values)}

\CommentTok{#Plot the values}
\KeywordTok{plot}\NormalTok{(fitvalues, standardized) }
\KeywordTok{abline}\NormalTok{(}\DecValTok{0}\NormalTok{,}\DecValTok{0}\NormalTok{)}
\KeywordTok{abline}\NormalTok{(}\DataTypeTok{v =} \DecValTok{0}\NormalTok{)}
\end{Highlighting}
\end{Shaded}

\includegraphics{07_lab-1_files/figure-latex/homogs-1.pdf}

\begin{Shaded}
\begin{Highlighting}[]
\CommentTok{# Alternative diagram}

\KeywordTok{plot}\NormalTok{(fake,}\DecValTok{1}\NormalTok{)}
\end{Highlighting}
\end{Shaded}

\includegraphics{07_lab-1_files/figure-latex/homogs-2.pdf}

\begin{Shaded}
\begin{Highlighting}[]
\CommentTok{#Homogeneity - }

\CommentTok{#The difference in the spread above and below the line is 1 and we meet the assumption of Homogeneity to some extent.}

\CommentTok{#The difference in the spread across the x axis is very close and even giving us the assumption of Homoscedasticity to a large extent.}
\end{Highlighting}
\end{Shaded}

\hypertarget{hypothesis-testing-graphs}{%
\section{Hypothesis Testing / Graphs:}\label{hypothesis-testing-graphs}}

Create a scatter plot of temporality and relativity.

\begin{verbatim}
a.  Be sure to check x/y axis labels and length.
b.  What type of relationship do these two variables appear to have?
\end{verbatim}

\begin{Shaded}
\begin{Highlighting}[]
\KeywordTok{library}\NormalTok{(ggplot2)}
\end{Highlighting}
\end{Shaded}

\begin{verbatim}
## Warning: package 'ggplot2' was built under R version 3.6.3
\end{verbatim}

\begin{Shaded}
\begin{Highlighting}[]
\KeywordTok{ggplot}\NormalTok{(no_miss, }\KeywordTok{aes}\NormalTok{(temporality, relativity)) }\OperatorTok{+}
\StringTok{  }\KeywordTok{geom_point}\NormalTok{() }\OperatorTok{+}
\StringTok{  }\KeywordTok{xlim}\NormalTok{(}\KeywordTok{min}\NormalTok{(no_miss}\OperatorTok{$}\NormalTok{temporality, }\DataTypeTok{na.rm =} \OtherTok{TRUE}\NormalTok{), }\KeywordTok{max}\NormalTok{(no_miss}\OperatorTok{$}\NormalTok{temporality, }\DataTypeTok{na.rm =} \OtherTok{TRUE}\NormalTok{)) }\OperatorTok{+}
\StringTok{  }\KeywordTok{ylim}\NormalTok{(}\KeywordTok{min}\NormalTok{(no_miss}\OperatorTok{$}\NormalTok{relativity, }\DataTypeTok{na.rm =} \OtherTok{TRUE}\NormalTok{), }\KeywordTok{max}\NormalTok{(no_miss}\OperatorTok{$}\NormalTok{relativity, }\DataTypeTok{na.rm =} \OtherTok{TRUE}\NormalTok{)) }\OperatorTok{+}
\StringTok{    }\KeywordTok{xlab}\NormalTok{(}\StringTok{"Temporality"}\NormalTok{) }\OperatorTok{+}
\StringTok{    }\KeywordTok{ylab}\NormalTok{(}\StringTok{"Relativity"}\NormalTok{)}
\end{Highlighting}
\end{Shaded}

\includegraphics{07_lab-1_files/figure-latex/plot1-1.pdf}

\begin{Shaded}
\begin{Highlighting}[]
\CommentTok{#These two variables do not seem to have any relationship.}
\end{Highlighting}
\end{Shaded}

Create a scatter plot of expectability and positive emotion.

\begin{verbatim}
a.  Include a linear line on the graph. 
b.  Be sure to check x/y axis labels and length.
c.  What type of relationship do these two variables appear to have?
\end{verbatim}

\begin{Shaded}
\begin{Highlighting}[]
\KeywordTok{ggplot}\NormalTok{(no_miss, }\KeywordTok{aes}\NormalTok{(expectability, positive)) }\OperatorTok{+}
\StringTok{  }\KeywordTok{geom_point}\NormalTok{() }\OperatorTok{+}
\StringTok{  }\KeywordTok{xlim}\NormalTok{(}\KeywordTok{min}\NormalTok{(no_miss}\OperatorTok{$}\NormalTok{expectability, }\DataTypeTok{na.rm =} \OtherTok{TRUE}\NormalTok{), }\KeywordTok{max}\NormalTok{(no_miss}\OperatorTok{$}\NormalTok{expectability, }\DataTypeTok{na.rm =} \OtherTok{TRUE}\NormalTok{)) }\OperatorTok{+}
\StringTok{  }\KeywordTok{ylim}\NormalTok{(}\KeywordTok{min}\NormalTok{(no_miss}\OperatorTok{$}\NormalTok{positive, }\DataTypeTok{na.rm =} \OtherTok{TRUE}\NormalTok{), }\KeywordTok{max}\NormalTok{(no_miss}\OperatorTok{$}\NormalTok{positive, }\DataTypeTok{na.rm =} \OtherTok{TRUE}\NormalTok{)) }\OperatorTok{+}
\StringTok{    }\KeywordTok{xlab}\NormalTok{(}\StringTok{"Expectability"}\NormalTok{) }\OperatorTok{+}
\StringTok{    }\KeywordTok{ylab}\NormalTok{(}\StringTok{"Positive"}\NormalTok{)}
\end{Highlighting}
\end{Shaded}

\includegraphics{07_lab-1_files/figure-latex/plot2-1.pdf}

\begin{Shaded}
\begin{Highlighting}[]
\CommentTok{#Thse two variables Expectability and Positive emotion do not appear to have any kind of relationship.}
\end{Highlighting}
\end{Shaded}

Create a scatter plot of expectability and relativity, grouping by
gender.

\begin{verbatim}
a.  Include a linear line on the graph. 
b.  Be sure to check x/y axis labels and length.
c.  What type of relationship do these two variables appear to have for each group?
\end{verbatim}

\begin{Shaded}
\begin{Highlighting}[]
\KeywordTok{ggplot}\NormalTok{(no_miss, }\KeywordTok{aes}\NormalTok{(expectability, relativity)) }\OperatorTok{+}
\StringTok{  }\KeywordTok{geom_point}\NormalTok{() }\OperatorTok{+}
\StringTok{    }\KeywordTok{geom_smooth}\NormalTok{(}\DataTypeTok{method =}\NormalTok{ lm, }\KeywordTok{aes}\NormalTok{(}\DataTypeTok{fill =}\NormalTok{ gender)) }\OperatorTok{+}
\StringTok{  }\KeywordTok{xlim}\NormalTok{(}\KeywordTok{min}\NormalTok{(no_miss}\OperatorTok{$}\NormalTok{expectability, }\DataTypeTok{na.rm =} \OtherTok{TRUE}\NormalTok{), }\KeywordTok{max}\NormalTok{(no_miss}\OperatorTok{$}\NormalTok{expectability, }\DataTypeTok{na.rm =} \OtherTok{TRUE}\NormalTok{)) }\OperatorTok{+}
\StringTok{  }\KeywordTok{ylim}\NormalTok{(}\KeywordTok{min}\NormalTok{(no_miss}\OperatorTok{$}\NormalTok{relativity, }\DataTypeTok{na.rm =} \OtherTok{TRUE}\NormalTok{), }\KeywordTok{max}\NormalTok{(no_miss}\OperatorTok{$}\NormalTok{relativity, }\DataTypeTok{na.rm =} \OtherTok{TRUE}\NormalTok{)) }\OperatorTok{+}
\StringTok{    }\KeywordTok{xlab}\NormalTok{(}\StringTok{"Expectability"}\NormalTok{) }\OperatorTok{+}
\StringTok{    }\KeywordTok{ylab}\NormalTok{(}\StringTok{"Relativity"}\NormalTok{)}
\end{Highlighting}
\end{Shaded}

\begin{verbatim}
## `geom_smooth()` using formula 'y ~ x'
\end{verbatim}

\includegraphics{07_lab-1_files/figure-latex/plot3-1.pdf}

\begin{Shaded}
\begin{Highlighting}[]
\CommentTok{#Expectability and Relativity do not seem to have any relationship to the gender.}
\end{Highlighting}
\end{Shaded}

Include a correlation table of all of the variables (cor).

\begin{verbatim}
a.  Include the output for Pearson.
b.  Include the output for Spearman.
c.  Include the output for Kendall.
d.  Which correlation was the strongest?
e.  For the correlations with gender, would point biserial or biserial be more appropriate?  Why?
\end{verbatim}

\begin{Shaded}
\begin{Highlighting}[]
\CommentTok{#a.}

\KeywordTok{cor}\NormalTok{(no_miss, }\DataTypeTok{use=}\StringTok{"pairwise.complete.obs"}\NormalTok{, }\DataTypeTok{method =} \StringTok{"pearson"}\NormalTok{)}
\end{Highlighting}
\end{Shaded}

\begin{verbatim}
##                   gender temporality expectability  relativity    positive
## gender         1.0000000   0.8211474    0.21596478  0.24314335 -0.28672829
## temporality    0.8211474   1.0000000    0.22107341  0.20824422 -0.25348617
## expectability  0.2159648   0.2210734    1.00000000  0.29778257 -0.01973344
## relativity     0.2431433   0.2082442    0.29778257  1.00000000 -0.02019202
## positive      -0.2867283  -0.2534862   -0.01973344 -0.02019202  1.00000000
\end{verbatim}

\begin{Shaded}
\begin{Highlighting}[]
\CommentTok{#b.}

\KeywordTok{cor}\NormalTok{(no_miss, }\DataTypeTok{use=}\StringTok{"pairwise.complete.obs"}\NormalTok{, }\DataTypeTok{method =} \StringTok{"spearman"}\NormalTok{)}
\end{Highlighting}
\end{Shaded}

\begin{verbatim}
##                   gender temporality expectability  relativity     positive
## gender         1.0000000   0.8292964   0.194344772 0.245107045 -0.271406084
## temporality    0.8292964   1.0000000   0.200975416 0.208455916 -0.252025278
## expectability  0.1943448   0.2009754   1.000000000 0.298590513  0.004897362
## relativity     0.2451070   0.2084559   0.298590513 1.000000000  0.005051013
## positive      -0.2714061  -0.2520253   0.004897362 0.005051013  1.000000000
\end{verbatim}

\begin{Shaded}
\begin{Highlighting}[]
\CommentTok{#c.}

\KeywordTok{cor}\NormalTok{(no_miss, }\DataTypeTok{use=}\StringTok{"pairwise.complete.obs"}\NormalTok{, }\DataTypeTok{method =} \StringTok{"kendall"}\NormalTok{)}
\end{Highlighting}
\end{Shaded}

\begin{verbatim}
##                   gender temporality expectability   relativity     positive
## gender         1.0000000   0.6788168   0.172909044  0.200631271 -0.222158232
## temporality    0.6788168   1.0000000   0.157765176  0.139231511 -0.165321557
## expectability  0.1729090   0.1577652   1.000000000  0.218075550  0.002423688
## relativity     0.2006313   0.1392315   0.218075550  1.000000000 -0.004212984
## positive      -0.2221582  -0.1653216   0.002423688 -0.004212984  1.000000000
\end{verbatim}

\begin{Shaded}
\begin{Highlighting}[]
\CommentTok{#d.}

\StringTok{"Correlation between gender and temporality is the strongest with all the three correlation methods."}
\end{Highlighting}
\end{Shaded}

\begin{verbatim}
## [1] "Correlation between gender and temporality is the strongest with all the three correlation methods."
\end{verbatim}

\begin{Shaded}
\begin{Highlighting}[]
\CommentTok{#e.}

\StringTok{"Point biserial is correlation with a true dichotomy variable, having no underlying continuum. Gender fits this status and hence for correlations with gender, point biserial would be more appropriate."}
\end{Highlighting}
\end{Shaded}

\begin{verbatim}
## [1] "Point biserial is correlation with a true dichotomy variable, having no underlying continuum. Gender fits this status and hence for correlations with gender, point biserial would be more appropriate."
\end{verbatim}

Calculate confidence interval for temporality and relativity.

\begin{Shaded}
\begin{Highlighting}[]
\KeywordTok{with}\NormalTok{(no_miss, }\KeywordTok{cor.test}\NormalTok{(temporality, relativity))}
\end{Highlighting}
\end{Shaded}

\begin{verbatim}
## 
##  Pearson's product-moment correlation
## 
## data:  temporality and relativity
## t = 2.9884, df = 197, p-value = 0.003162
## alternative hypothesis: true correlation is not equal to 0
## 95 percent confidence interval:
##  0.07121707 0.33755692
## sample estimates:
##       cor 
## 0.2082442
\end{verbatim}

Calculate the difference in correlations for 1) temporality and
expectbility and 2) temporality and positive emotion.

\begin{verbatim}
a.  Include the output from the test through Pearson's test.
b.  Is there a significant difference in their correlations?
\end{verbatim}

\begin{Shaded}
\begin{Highlighting}[]
\CommentTok{#a)}

\KeywordTok{cor}\NormalTok{(no_miss}\OperatorTok{$}\NormalTok{temporality,no_miss}\OperatorTok{$}\NormalTok{expectability, }\DataTypeTok{use=}\StringTok{"pairwise.complete.obs"}\NormalTok{, }\DataTypeTok{method =} \StringTok{"pearson"}\NormalTok{)}
\end{Highlighting}
\end{Shaded}

\begin{verbatim}
## [1] 0.2210734
\end{verbatim}

\begin{Shaded}
\begin{Highlighting}[]
\KeywordTok{cor}\NormalTok{(no_miss}\OperatorTok{$}\NormalTok{temporality,no_miss}\OperatorTok{$}\NormalTok{positive, }\DataTypeTok{use=}\StringTok{"pairwise.complete.obs"}\NormalTok{, }\DataTypeTok{method =} \StringTok{"pearson"}\NormalTok{)}
\end{Highlighting}
\end{Shaded}

\begin{verbatim}
## [1] -0.2534862
\end{verbatim}

\begin{Shaded}
\begin{Highlighting}[]
\KeywordTok{library}\NormalTok{(}\StringTok{"cocor"}\NormalTok{)}
\end{Highlighting}
\end{Shaded}

\begin{verbatim}
## Warning: package 'cocor' was built under R version 3.6.3
\end{verbatim}

\begin{Shaded}
\begin{Highlighting}[]
\KeywordTok{cocor}\NormalTok{(}\OperatorTok{~}\NormalTok{temporality }\OperatorTok{+}\StringTok{ }\NormalTok{expectability }\OperatorTok{|}\StringTok{ }\NormalTok{temporality }\OperatorTok{+}\StringTok{ }\NormalTok{positive, }\DataTypeTok{data =}\NormalTok{ no_miss)}
\end{Highlighting}
\end{Shaded}

\begin{verbatim}
## 
##   Results of a comparison of two overlapping correlations based on dependent groups
## 
## Comparison between r.jk (temporality, expectability) = 0.2211 and r.jh (temporality, positive) = -0.2535
## Difference: r.jk - r.jh = 0.4746
## Related correlation: r.kh = -0.0197
## Data: no_miss: j = temporality, k = expectability, h = positive
## Group size: n = 199
## Null hypothesis: r.jk is equal to r.jh
## Alternative hypothesis: r.jk is not equal to r.jh (two-sided)
## Alpha: 0.05
## 
## pearson1898: Pearson and Filon's z (1898)
##   z = 5.0382, p-value = 0.0000
##   Null hypothesis rejected
## 
## hotelling1940: Hotelling's t (1940)
##   t = 4.9340, df = 196, p-value = 0.0000
##   Null hypothesis rejected
## 
## williams1959: Williams' t (1959)
##   t = 4.9336, df = 196, p-value = 0.0000
##   Null hypothesis rejected
## 
## olkin1967: Olkin's z (1967)
##   z = 5.0382, p-value = 0.0000
##   Null hypothesis rejected
## 
## dunn1969: Dunn and Clark's z (1969)
##   z = 4.8104, p-value = 0.0000
##   Null hypothesis rejected
## 
## hendrickson1970: Hendrickson, Stanley, and Hills' (1970) modification of Williams' t (1959)
##   t = 4.9336, df = 196, p-value = 0.0000
##   Null hypothesis rejected
## 
## steiger1980: Steiger's (1980) modification of Dunn and Clark's z (1969) using average correlations
##   z = 4.7437, p-value = 0.0000
##   Null hypothesis rejected
## 
## meng1992: Meng, Rosenthal, and Rubin's z (1992)
##   z = 4.6799, p-value = 0.0000
##   Null hypothesis rejected
##   95% confidence interval for r.jk - r.jh: 0.2813 0.6866
##   Null hypothesis rejected (Interval does not include 0)
## 
## hittner2003: Hittner, May, and Silver's (2003) modification of Dunn and Clark's z (1969) using a backtransformed average Fisher's (1921) Z procedure
##   z = 4.7436, p-value = 0.0000
##   Null hypothesis rejected
## 
## zou2007: Zou's (2007) confidence interval
##   95% confidence interval for r.jk - r.jh: 0.2834 0.6535
##   Null hypothesis rejected (Interval does not include 0)
\end{verbatim}

\begin{Shaded}
\begin{Highlighting}[]
\CommentTok{#b)}

\CommentTok{#The difference of 0.4746 signifies moderate difference in their correlation, but not significant.}
\end{Highlighting}
\end{Shaded}

Calculate the difference in correlations for gender on temporality and
relativity.

\begin{verbatim}
a.  Include the output from the test.
b.  Is there a significant difference in their correlations?
\end{verbatim}

\begin{Shaded}
\begin{Highlighting}[]
\CommentTok{#a)}

\KeywordTok{cor}\NormalTok{(no_miss}\OperatorTok{$}\NormalTok{gender, no_miss}\OperatorTok{$}\NormalTok{temporality, }\DataTypeTok{use=}\StringTok{"pairwise.complete.obs"}\NormalTok{, }\DataTypeTok{method =} \StringTok{"pearson"}\NormalTok{)}
\end{Highlighting}
\end{Shaded}

\begin{verbatim}
## [1] 0.8211474
\end{verbatim}

\begin{Shaded}
\begin{Highlighting}[]
\KeywordTok{cor}\NormalTok{(no_miss}\OperatorTok{$}\NormalTok{gender, no_miss}\OperatorTok{$}\NormalTok{relativity, }\DataTypeTok{use=}\StringTok{"pairwise.complete.obs"}\NormalTok{, }\DataTypeTok{method =} \StringTok{"pearson"}\NormalTok{)}
\end{Highlighting}
\end{Shaded}

\begin{verbatim}
## [1] 0.2431433
\end{verbatim}

\begin{Shaded}
\begin{Highlighting}[]
\KeywordTok{cocor}\NormalTok{(}\OperatorTok{~}\NormalTok{gender }\OperatorTok{+}\StringTok{ }\NormalTok{temporality }\OperatorTok{|}\StringTok{ }\NormalTok{gender }\OperatorTok{+}\StringTok{ }\NormalTok{relativity, }\DataTypeTok{data =}\NormalTok{ no_miss)}
\end{Highlighting}
\end{Shaded}

\begin{verbatim}
## 
##   Results of a comparison of two overlapping correlations based on dependent groups
## 
## Comparison between r.jk (gender, temporality) = 0.8211 and r.jh (gender, relativity) = 0.2431
## Difference: r.jk - r.jh = 0.578
## Related correlation: r.kh = 0.2082
## Data: no_miss: j = gender, k = temporality, h = relativity
## Group size: n = 199
## Null hypothesis: r.jk is equal to r.jh
## Alternative hypothesis: r.jk is not equal to r.jh (two-sided)
## Alpha: 0.05
## 
## pearson1898: Pearson and Filon's z (1898)
##   z = 8.4784, p-value = 0.0000
##   Null hypothesis rejected
## 
## hotelling1940: Hotelling's t (1940)
##   t = 11.3628, df = 196, p-value = 0.0000
##   Null hypothesis rejected
## 
## williams1959: Williams' t (1959)
##   t = 10.2578, df = 196, p-value = 0.0000
##   Null hypothesis rejected
## 
## olkin1967: Olkin's z (1967)
##   z = 8.4784, p-value = 0.0000
##   Null hypothesis rejected
## 
## dunn1969: Dunn and Clark's z (1969)
##   z = 9.5639, p-value = 0.0000
##   Null hypothesis rejected
## 
## hendrickson1970: Hendrickson, Stanley, and Hills' (1970) modification of Williams' t (1959)
##   t = 11.3609, df = 196, p-value = 0.0000
##   Null hypothesis rejected
## 
## steiger1980: Steiger's (1980) modification of Dunn and Clark's z (1969) using average correlations
##   z = 9.3551, p-value = 0.0000
##   Null hypothesis rejected
## 
## meng1992: Meng, Rosenthal, and Rubin's z (1992)
##   z = 9.1993, p-value = 0.0000
##   Null hypothesis rejected
##   95% confidence interval for r.jk - r.jh: 0.7179 1.1066
##   Null hypothesis rejected (Interval does not include 0)
## 
## hittner2003: Hittner, May, and Silver's (2003) modification of Dunn and Clark's z (1969) using a backtransformed average Fisher's (1921) Z procedure
##   z = 9.1969, p-value = 0.0000
##   Null hypothesis rejected
## 
## zou2007: Zou's (2007) confidence interval
##   95% confidence interval for r.jk - r.jh: 0.4467 0.7151
##   Null hypothesis rejected (Interval does not include 0)
\end{verbatim}

\begin{Shaded}
\begin{Highlighting}[]
\CommentTok{#b)}

\CommentTok{#The difference of 0.578 signifies moderate difference in their correlation, but not significant.}
\end{Highlighting}
\end{Shaded}

Calculate the partial and semipartial correlations for all variables,
and include the output. a. Are any of the correlations significant after
controlling for all other relationships?

\begin{Shaded}
\begin{Highlighting}[]
\CommentTok{#install.packages("ppcor")}
\KeywordTok{library}\NormalTok{(}\StringTok{"ppcor"}\NormalTok{)}
\end{Highlighting}
\end{Shaded}

\begin{verbatim}
## Warning: package 'ppcor' was built under R version 3.6.3
\end{verbatim}

\begin{verbatim}
## Loading required package: MASS
\end{verbatim}

\begin{Shaded}
\begin{Highlighting}[]
\KeywordTok{pcor}\NormalTok{(no_miss, }\DataTypeTok{method =} \StringTok{"pearson"}\NormalTok{)}
\end{Highlighting}
\end{Shaded}

\begin{verbatim}
## $estimate
##                    gender temporality expectability  relativity    positive
## gender         1.00000000  0.79215963    0.03391939  0.12228249 -0.14913195
## temporality    0.79215963  1.00000000    0.07829051 -0.00342090 -0.03650384
## expectability  0.03391939  0.07829051    1.00000000  0.25667189  0.03519006
## relativity     0.12228249 -0.00342090    0.25667189  1.00000000  0.04297327
## positive      -0.14913195 -0.03650384    0.03519006  0.04297327  1.00000000
## 
## $p.value
##                     gender  temporality expectability   relativity   positive
## gender        0.000000e+00 1.797635e-43  0.6369482700 0.0877443604 0.03696294
## temporality   1.797635e-43 0.000000e+00  0.2753910929 0.9620458730 0.61148549
## expectability 6.369483e-01 2.753911e-01  0.0000000000 0.0002818333 0.62437362
## relativity    8.774436e-02 9.620459e-01  0.0002818333 0.0000000000 0.54980391
## positive      3.696294e-02 6.114855e-01  0.6243736176 0.5498039074 0.00000000
## 
## $statistic
##                   gender temporality expectability relativity   positive
## gender         0.0000000  18.0784116     0.4727145  1.7160766 -2.1006588
## temporality   18.0784116   0.0000000     1.0938180 -0.0476479 -0.5087787
## expectability  0.4727145   1.0938180     0.0000000  3.6989456  0.4904445
## relativity     1.7160766  -0.0476479     3.6989456  0.0000000  0.5991019
## positive      -2.1006588  -0.5087787     0.4904445  0.5991019  0.0000000
## 
## $n
## [1] 199
## 
## $gp
## [1] 3
## 
## $method
## [1] "pearson"
\end{verbatim}

\begin{Shaded}
\begin{Highlighting}[]
\KeywordTok{spcor}\NormalTok{(no_miss, }\DataTypeTok{method =} \StringTok{"pearson"}\NormalTok{)}
\end{Highlighting}
\end{Shaded}

\begin{verbatim}
## $estimate
##                    gender  temporality expectability  relativity    positive
## gender         1.00000000  0.726030568    0.01898425  0.06891778 -0.08436264
## temporality    0.73797712  1.000000000    0.04465064 -0.00194503 -0.02076881
## expectability  0.03189244  0.073796190    1.00000000  0.24955531  0.03308863
## relativity     0.11532455 -0.003202056    0.24857821  1.00000000  0.04026112
## positive      -0.14410849 -0.034903017    0.03364526  0.04109933  1.00000000
## 
## $p.value
##                     gender  temporality expectability   relativity  positive
## gender        0.000000e+00 2.214962e-33  0.7917002840 0.3371462115 0.2397470
## temporality   5.633521e-35 0.000000e+00  0.5343229323 0.9784148496 0.7726314
## expectability 6.572225e-01 3.039778e-01  0.0000000000 0.0004196575 0.6452264
## relativity    1.074871e-01 9.644723e-01  0.0004428427 0.0000000000 0.5752902
## positive      4.388946e-02 6.272049e-01  0.6396748967 0.5673545180 0.0000000
## 
## $statistic
##                   gender temporality expectability  relativity   positive
## gender         0.0000000  14.7055230     0.2644677  0.96220142 -1.1792394
## temporality   15.2318316   0.0000000     0.6225323 -0.02709119 -0.2893384
## expectability  0.4444364   1.0306723     0.0000000  3.58947223  0.4611238
## relativity     1.6170745  -0.0445997     3.5744906  0.00000000  0.5612275
## positive      -2.0283714  -0.4864392     0.4688898  0.57293151  0.0000000
## 
## $n
## [1] 199
## 
## $gp
## [1] 3
## 
## $method
## [1] "pearson"
\end{verbatim}

\begin{Shaded}
\begin{Highlighting}[]
\StringTok{"The correlation between gender and temporality are significantly high for both partial and semipartial correlations."}
\end{Highlighting}
\end{Shaded}

\begin{verbatim}
## [1] "The correlation between gender and temporality are significantly high for both partial and semipartial correlations."
\end{verbatim}

\hypertarget{theory}{%
\section{Theory:}\label{theory}}

\begin{verbatim}
- What are we using as our model for understanding the data in a correlational analysis?

We are using Pearson's model for understanding the data in a correlational analysis.


- How might we determine model fit?
By using the correlation coefficient, we get r which we can look to determine the model fit or the strength of the relationship.

- What is the difference between correlation and covariance?
Correlation tells us the measure of relationship between two variables. It is the the extent to which they are related.

Covariance tells us by how much scores on two variables differ from their respective means.

- What is the difference between R and r?
R = correlation coefficient for 3+ variables
r = correlation coefficient for 2 variables

- When would I want to use a nonparametric correlation over Pearson's correlation?

Pearson’s correlation is used on the ranked data. We use nonparametric correlation when the data is not ranked.

- What is the distinction between semi-partial and partial correlations? 
A partial correlation measures the relationship between two variables, controlling for the effect that a third
variable has on them both.

A semi-partial correlation measures the relationship between two variables controlling for the effect that a third
variable has on only one of the others.
\end{verbatim}

\end{document}
